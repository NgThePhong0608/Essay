% MỞ ĐẦU (LỜI MỞ ĐẦU - Chương 0)

\chapter*{MỞ ĐẦU} % Tên của chương
\addcontentsline{toc}{chapter}{MỞ ĐẦU} % Thêm tên chương vào mục lục

\label{Chapter0} % Để trích dẫn chương này ở chỗ nào đó trong bài, hãy sử dụng lệnh \ref{Chapter0} 

%----------------------------------------------------------------------------------------

Xử lý ngôn ngữ tự nhiên (NLP) là một lĩnh vực học máy đang ngày càng phổ biến trong các lĩnh vực như AI, Robotics, Data science, ... và ngày càng có tầm ảnh hưởng lớn hơn trong cuộc cách mạng 4.0 của thế giới.

Bài nghiên cứu này trình bày về xử lý ngôn ngữ tự nhiên và các thuật toán cho đề tài "Nghiên cứu phân tích văn bản trong học máy". Nội dung của bài nghiên cứu này gồm 4 chương và 1 phụ lục:

\begin{enumerate}
	\item \textbf{Chương \ref{Chapter1}: Xử lý ngôn ngữ tự nhiên - Natural language proccessing}\\
	Giới thiệu về NLP và các framework được sử dụng rộng rãi trong lĩnh vực này
	\item \textbf{Chương \ref{Chapter2}: Feature Engineering và các model xử lý dữ liệu dạng văn bản}\\
	Giới thiệu về Feature Engineering (Kĩ thuật phổ biến trong các bài toán về học máy) và các model đã và đang được áp dụng vào NLP
	\item \textbf{Chương \ref{Chapter3}: Topic Modeling}
	Một bài toán đặc trưng của xử lý dữ liệu dạng văn bản \\
	Các thuật toán như \textbf{LSA và LDA}
	\item \textbf{Chương \ref{Chapter4}: Các model nâng cao trong xử lý dữ liệu dạng văn bản}\\
	Các model giải quyết các bài toán phức tạp hơn của việc xử lý ngôn ngữ và là sự cải tiến của các model được giới thiệu ở Chương \ref{Chapter2}
	\item \textbf{Phụ lục \ref{AppendixB}: Liệt kê source code} 
\end{enumerate}

