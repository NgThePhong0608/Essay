%%%%%%%%%%%%%%%%%%%%%%%%%%%%%%%%%%%%%%%%%%%%%%%%%%%%%%%%%%%%%%%%%%%%%%%%%%%%%%%%%%%%%%%%%
% Khóa luận tốt nghiệp/Tiểu luận 
% LaTeX Template
% Phiên bản 1.0 (tạo ra ngày 05/10/2022)
%
% Template này có thể được download tại:
% https://github.com/cpc1996/HUS-Dissertation-Template
%
% Phiên bản 1.0 được chỉnh sửa bởi:
% Công Phương Cao (congphuongcao@gmail.com)
% Nguyễn Cảnh Việt (vietncp@gmail.com)
%
% Template được tham khảo từ một phiên bản template của:
% Steve Gunn (http://users.ecs.soton.ac.uk/srg/softwaretools/document/templates/)
% Sunil Patel (http://www.sunilpatel.co.uk/thesis-template/)
%
%%%%%%%%%%%%%%%%%%%%%%%%%%%%%%%%%%%%%%%%%%%%%%%%%%%%%%%%%%%%%%%%%%%%%%%%%%%%%%%%%%%%%%%%%


%----------------------------------------------------------------------------------------
%	(KHÔNG CHỈNH SỬA PHẦN NÀY)
%
%	PHẦN 1: CÁC PACKAGE CƠ BẢN VÀ CÁC TÙY CHỈNH VĂN BẢN
%----------------------------------------------------------------------------------------

\documentclass[
12pt,
oneside,
english,
doublespacing,
nolistspacing,
liststotoc,
parskip,
headsepline,
chapterinoneline,
]{HUSdissertation}

% \usepackage[utf8]{inputenc} 
\usepackage[utf8]{vietnam} 
%\usepackage[T1]{fontenc}

\usepackage{mathptmx}
\usepackage{amsmath}
\allowdisplaybreaks

%https://www.overleaf.com/learn/latex/Biblatex_citation_styles
%\usepackage[backend=bibtex,style=authoryear,natbib=true]{biblatex}
\usepackage[backend=bibtex,style=numeric,citestyle=ieee,natbib=true]{biblatex}

\addbibresource{main.bib}

\usepackage[autostyle=true]{csquotes}


%----------------------------------------------------------------------------------------
%	PHẦN 2: CÁC PACKAGE BỔ TRỢ THÊM VÀO TRONG QUÁ TRÌNH BIÊN SOẠN
%----------------------------------------------------------------------------------------

\RequirePackage{setlst}		% Liệt kê/trích dẫn code

\usepackage{multirow}
\usepackage{subfigure}
\usepackage[fontsize=13pt]{scrextend}

%https://www.sascha-frank.com/latex-font-size.html
%https://tex.stackexchange.com/questions/103286/how-to-change-section-subsection-font-size
\usepackage{titlesec}

\titleformat{\section}
{\normalfont\fontsize{13}{13}\bfseries}{\thesection}{1em}{}

\titleformat{\subsection}
{\normalfont\fontsize{13}{13}\bfseries\itshape}{\thesubsection}{1em}{}

%https://tex.stackexchange.com/questions/351961/how-to-indent-code-in-beginverbatim
\usepackage{fancyvrb} 		% Fancy Verbatim
\fvset{tabsize=4,vspace=0pt,fontsize=\footnotesize}

\usepackage{longtable} 		% Bảng dài - Long table

\usepackage[figuresright]{rotating} % Bảng ngang - Sideways table
\usepackage{tabularx}

\usepackage{fontawesome5} 	% Các biểu tượng, ký hiệu đặc biệt

\usepackage{tikz} 			% Vẽ hình
\usetikzlibrary{calc}

\usepackage{indentfirst}	% Lùi đầu dòng ở đầu đoạn văn
\setlength{\parindent}{0.5cm}

%----------------------------------------------------------------------------------------
%	PHẦN 3: THÔNG TIN VỀ Khóa luận tốt nghiệp/Tiểu luận (THESIS INFORMATION)
%----------------------------------------------------------------------------------------

\author{NGUYỄN THẾ PHONG} 			% Ví dụ: "Nguyễn Thị Thu Thảo"
\thesistitle{NGUYÊN CỨU PHÂN TÍCH VĂN BẢN TRONG HỌC MÁY} % Ví dụ: "Nghiên cứu chế tạo và tính chất quang học của vật liệu nano LaF3:Sm3+"

\supervisor {TS. PHẠM TIẾN LÂM} 	% Ví dụ: "PGS. TS. Nguyễn Ngọc Long"
\supervisorr{NGUYỄN VĂN QUYỀN\\
	ĐẶNG VĂN BÁU}

\field{Ngành Kỹ thuật điện tử và tin học} 					% Ví dụ: "Ngành Vật lý", "Ngành Hóa học", v.v.
\program{Chương trình đào tạo chuẩn} 			% Ví dụ: "Chương trình đào tạo chuẩn", "Chương trình đào tạo cử nhân tài năng", v.v.
\doctype{Tiểu luận} % Điền vào đây: "Khóa luận tốt nghiệp đại học hệ chính quy" hoặc "Tiểu luận"

\university{\href{http://hus.vnu.edu.vn}{Trường Đại học Khoa học Tự nhiên}}
\department{\href{http://hus.vnu.edu.vn/gioi-thieu/co-cau-to-chuc/khoa-truc-thuoc/khoa-vat-ly.html}{Khoa Vật lý}}

\AtBeginDocument{
	\hypersetup{pdftitle=\ttitle}
	\hypersetup{pdfauthor=\authorname}
}


\begin{document}

\lstset{style=codeC}	% Thiết lập ngôn ngữ C/C++ là ngôn ngữ mặc định cho phần liệt kê souce code của cả bài

\frontmatter 			% Sử dụng hệ thống đánh số La Mã (i, ii, iii, iv...) cho những trang trước phần mục lục

\pagestyle{plain} 


%----------------------------------------------------------------------------------------
%	(KHÔNG CHỈNH SỬA PHẦN NÀY)
%
%	PHẦN 4: TRANG TIÊU ĐỀ/TRANG BÌA (TITLE PAGE)
%----------------------------------------------------------------------------------------

% TRANG BÌA CHÍNH:
\include{main-cover}

% TRANG BÌA PHỤ:
\include{sub-cover}


%----------------------------------------------------------------------------------------
%	PHẦN 5: DANH NGÔN (QUOTES)
%----------------------------------------------------------------------------------------

%\vspace*{0.2\textheight}
%
%\noindent\enquote{\itshape 
%	Cái tôi và sự hiểu biết tỷ lệ nghịch với nhau. Hiểu biết càng nhiều cái tôi càng bé. Hiểu biết càng ít, cái tôi càng to.
%}\bigbreak
%
%\hfill Albert Einstein


%----------------------------------------------------------------------------------------
%	PHẦN 6: LỜI CẢM ƠN (ACKNOWLEDGEMENTS)
%----------------------------------------------------------------------------------------

\begin{acknowledgements}
	\addchaptertocentry{\acknowledgementname}
	\thispagestyle{empty}
	Lời đầu tiên em xin phép được nói lời cảm ơn Bố Mẹ và người thân trong gia đình đã
	có công sinh thành và nuôi dưỡng, dạy bảo để em có được như ngày hôm nay. Em cũng
	xin gửi lời cảm ơn đến toàn thể các Thầy, Cô, bạn bè, anh, chị, em...v.v. có người đến, có
	người đi và có người ở lại, mọi người đều để lại cho em những bài học, kiến thức giúp
	em trưởng thành hơn \\
	Tiếp đó em xin gửi lời cảm ơn đến toàn thể các quý Thầy, Cô đang công tác và giảng
	dạy tại Trường Đại Học Khoa Học Tự Nhiên - Đại Học Quốc Gia Hà Nội. Đặc biệt là
	các quý thầy cô khoa Vật Lý nói chung và bộ môn Tin Học Vật Lý - Điện Tử Vô Tuyến
	nói riêng, các Thầy, Cô rất tận tâm với nghề, nhiệt tình chỉ bảo truyền đạt kiến thức và
	những kinh nghiệm quý báu tích lũy được trong hành trình trồng người của mình đến
	với chúng em, đó là những món quà vô giá em rất may mắn được đón nhận. Đặc biệt em xin được gửi lời cảm ơn tới Thầy Phạm Tiến Lâm - Giảng viên hướng dẫn đã tận tình hướng dẫn để em có thể hoàn thành đề tài một cách tốt đẹp \\
	Do kiến thức còn hạn chế vì vậy còn nhiều sai sót, em kính mong nhận được lời góp ý của các thầy để em có thể hoàn thành tốt hơn ở những đề tài sau.

\end{acknowledgements}


%----------------------------------------------------------------------------------------
%	(KHÔNG CHỈNH SỬA PHẦN NÀY)
%
%	PHẦN 7: MỤC LỤC (LIST OF CONTENTS/FIGURES/TABLES PAGES)
%----------------------------------------------------------------------------------------

\begin{spacing}{1.15}
	\tableofcontents 	% In ra mục lục chính
\end{spacing}

\begin{spacing}{1.15}
	\listoffigures 		% In ra danh sách hình vẽ
\end{spacing}

%\begin{spacing}{1.15}
%	\listoftables		% In ra danh sách bảng
%\end{spacing}


%----------------------------------------------------------------------------------------
%	PHẦN 8: DANH SÁCH TÊN VIẾT TẮT (ABBREVIATIONS)
%----------------------------------------------------------------------------------------

\begin{abbreviations}{ll} % Thêm danh sách tên viết tắt (dưới dạng một bảng có 2 cột)
\textbf{AI} & \textbf{A}rtificial \textbf{I}ntelligence\\
\textbf{NLP} & \textbf{N}atural\textbf{L}anguage \textbf{P}rocessing\\
\textbf{IE} & \textbf{I}nformation \textbf{E}xtraction\\
\textbf{IE} & \textbf{I}nformation \textbf{R}etrieval\\
\textbf{MT} & \textbf{M}achine\textbf{T}ranslation\\
\textbf{CRISP-DM} & \textbf{C}ross \textbf{I}ndustry \textbf{S}tandard \textbf{P}rocess \textbf{F}or \textbf{D}ata \textbf{M}ining\\
\textbf{TF-IDF} & \textbf{T}erm \textbf{F}requency \textbf{I}nverse \textbf{D}ocument \textbf{F}requency \\
\textbf{LSA} & \textbf{L}atent \textbf{S}emantic \textbf{A}nalysis \\
\textbf{LDA} & \textbf{L}atent \textbf{D}irichlet \textbf{A}llocation \\
\end{abbreviations}


%----------------------------------------------------------------------------------------
%	PHẦN 9: CÁC HẰNG SỐ VẬT LÝ/THÔNG SỐ KĨ THUẬT (PHYSICAL CONSTANTS/OTHER DEFINITIONS)
%----------------------------------------------------------------------------------------

%\begin{constants}{lr@{${}={}$}l} % Thêm danh sách các hằng số (dưới dạng một bảng có 3 cột)
%
%%	Lệnh \SI{}{} được cung cấp bởi gói siunitx, hãy đọc tài liệu hướng dẫn để biết cách sử dụng nó
%
%%	Tên hằng số 	 & $Biểu tượng$	& $Hằng số$ cùng với đơn vị\\
%	Vận tốc ánh sáng & $c_{0}$		& \SI{2.99792458e8}{\meter\per\second} (chính xác)\\
%
%\end{constants}


%----------------------------------------------------------------------------------------
%	PHẦN 10: DANH SÁCH KÝ HIỆU (SYMBOLS)
%----------------------------------------------------------------------------------------

%\begin{symbols}{lll} % Thêm danh sách các ký hiệu (dưới dạng một bảng có 3 cột)
%
%%   Ký hiệu	& Ý nghĩa		& Đơn vị \\
%	$a$		& khoảng cách	& \si{\meter} \\
%	$P$		& công suất		& \si{\watt} (\si{\joule\per\second}) \\
%
%	\addlinespace % Khoảng cách để phân biệt giữa ký hiệu Latin với ký hiệu La Mã
%
%	$\omega$ & tần số góc	& \si{\radian} \\
%
%\end{symbols}


%----------------------------------------------------------------------------------------
%	(KHÔNG CHỈNH SỬA PHẦN NÀY)
%
%	PHẦN 11: LỜI ĐỀ TẶNG (DEDICATION)
%----------------------------------------------------------------------------------------

%\dedicatory{Dành tặng/Dành cho/Gửi tới\ldots} 


%----------------------------------------------------------------------------------------
%	PHẦN 12: NỘI DUNG/CÁC CHƯƠNG Khóa luận tốt nghiệp/Tiểu luận (THESIS CONTENT - CHAPTERS)
%----------------------------------------------------------------------------------------

\mainmatter % Bắt đầu đánh số trang (1,2,3...)

\pagestyle{plain}

% Hãy thêm những chương (chapter) của khóa luận/tiểu luận vào thư mục Chapters
% Hãy bỏ chú thích những dòng nếu bạn đã bổ sung những chương vào

% MỞ ĐẦU (LỜI MỞ ĐẦU - Chương 0)

\chapter*{MỞ ĐẦU} % Tên của chương
\addcontentsline{toc}{chapter}{MỞ ĐẦU} % Thêm tên chương vào mục lục

\label{Chapter0} % Để trích dẫn chương này ở chỗ nào đó trong bài, hãy sử dụng lệnh \ref{Chapter0} 

%----------------------------------------------------------------------------------------

Xử lý ngôn ngữ tự nhiên (NLP) là một lĩnh vực học máy đang ngày càng phổ biến trong các lĩnh vực như AI, Robotics, Data science, ... và ngày càng có tầm ảnh hưởng lớn hơn trong cuộc cách mạng 4.0 của thế giới.

Bài nghiên cứu này trình bày về xử lý ngôn ngữ tự nhiên và các thuật toán cho đề tài "Nghiên cứu phân tích văn bản trong học máy". Nội dung của bài nghiên cứu này gồm 4 chương và 1 phụ lục:

\begin{enumerate}
	\item \textbf{Chương \ref{Chapter1}: Xử lý ngôn ngữ tự nhiên - Natural language proccessing}\\
	Giới thiệu về NLP và các framework được sử dụng rộng rãi trong lĩnh vực này
	\item \textbf{Chương \ref{Chapter2}: Feature Engineering và các model xử lý dữ liệu dạng văn bản}\\
	Giới thiệu về Feature Engineering (Kĩ thuật phổ biến trong các bài toán về học máy) và các model đã và đang được áp dụng vào NLP
	\item \textbf{Chương \ref{Chapter3}: Topic Modeling}
	Một bài toán đặc trưng của xử lý dữ liệu dạng văn bản \\
	Các thuật toán như \textbf{LSA và LDA}
	\item \textbf{Chương \ref{Chapter4}: Các model nâng cao trong xử lý dữ liệu dạng văn bản}\\
	Các model giải quyết các bài toán phức tạp hơn của việc xử lý ngôn ngữ và là sự cải tiến của các model được giới thiệu ở Chương \ref{Chapter2}
	\item \textbf{Phụ lục \ref{AppendixB}: Liệt kê source code} 
\end{enumerate}


%\include{Chapters/Chapter1}
% Chương 1

\chapter{Xử lý ngôn ngữ tự nhiên - Natural language proccessing} % Tên của chương

\label{Chapter1} % Để trích dẫn chương này ở chỗ nào đó trong bài, hãy sử dụng lệnh \ref{Chapter1} 

%----------------------------------------------------------------------------------------

% Định nghĩa một số lệnh cần thiết để điều chỉnh định dạng cho một số nội dung nhất định trong bài
\newcommand{\keyword}[1]{\textbf{#1}}
\newcommand{\tabhead}[1]{\textbf{#1}}
\newcommand{\code}[1]{\texttt{#1}}
\newcommand{\file}[1]{\texttt{\bfseries#1}}
\newcommand{\option}[1]{\texttt{\itshape#1}}

%----------------------------------------------------------------------------------------

\section{Giới thiệu về xử lí ngôn ngữ tự nhiên (NLP)}

Xử lý ngôn ngữ tự nhiên là một nhánh của Trí tuệ nhân tạo, tập trung vào việc nghiên cứu sự tương tác giữa máy tính và ngôn ngữ tự nhiên của con người, dưới dạng tiếng nói (speech) hoặc văn bản (text)\cite{WEBSITE:1}.

Mục tiêu của lĩnh vực này là giúp máy tính hiểu và thực hiện hiệu quả những nhiệm vụ liên quan đến ngôn ngữ của con người như: tương tác giữa người và máy, cải thiện hiệu quả giao tiếp giữa con người với con người, hoặc đơn giản là nâng cao hiệu quả xử lý văn bản và lời nói.



%----------------------------------------------------------------------------------------

\section{Lịch sử hình thành và phát triển của NLP}

Xử lý ngôn ngữ tự nhiên ra đời từ những năm 40 của thế kỷ 20, trải qua các giai đoạn phát triển với nhiều phương pháp và mô hình xử lý khác nhau. Có thể kể tới các phương pháp sử dụng ô-tô-mát và mô hình xác suất (những năm 50), các phương pháp dựa trên ký hiệu, các phương pháp ngẫu nhiên (những năm 70), các phương pháp sử dụng học máy truyền thống (những năm đầu thế kỷ 21), và đặc biệt là sự bùng nổ của học sâu trong thập kỷ vừa qua\cite{WEBSITE:1}.

Xử lý ngôn ngữ tự nhiên có thể được chia ra thành hai nhánh lớn, không hoàn toàn độc lập, bao gồm xử lý tiếng nói (speech processing) và xử lý văn bản (text processing).

Xử lý tiếng nói tập trung nghiên cứu, phát triển các thuật toán, chương trình máy tính xử lý ngôn ngữ của con người ở dạng tiếng nói (dữ liệu âm thanh). Các ứng dụng quan trọng của xử lý tiếng nói bao gồm nhận dạng tiếng nói và tổng hợp tiếng nói.

Xử lý văn bản tập trung vào phân tích dữ liệu văn bản. Các ứng dụng quan trọng của xử lý văn bản bao gồm tìm kiếm và truy xuất thông tin, dịch máy, tóm tắt văn bản tự động, hay kiểm lỗi chính tả tự động. Xử lý văn bản đôi khi được chia tiếp thành hai nhánh nhỏ hơn bao gồm hiểu văn bản và sinh văn bản. Nếu như hiểu liên quan tới các bài toán phân tích văn bản thì sinh liên quan tới nhiệm vụ tạo ra văn bản mới như trong các ứng dụng về dịch máy hoặc tóm tắt văn bản tự động.

\section{Giao tiếp giữa người và máy dựa trên NLP}
Ngày nay, nhiều hệ thống/chương trình máy tính có khả năng giao tiếp với con người thông qua ngôn ngữ tự nhiên, hoặc dưới dạng văn bản, hoặc dưới dạng tiếng nói. Các ứng dụng tiêu biểu giao tiếp dưới dạng văn bản có thể kể đến như tìm kiếm thông tin, chatbot, dịch máy. Các ứng dụng giao tiếp qua tiếng nói như trợ lý ảo, tìm kiếm bằng giọng nói (điện thoại, tivi), và điều khiển qua giọng nói (điện thoại, các thiết bị gia đình)\cite{WEBSITE:1}.

Hình \ref{pic1.1} mô tả kiến trúc tiêu biểu của một chương trình máy tính giao tiếp với con người qua tiếng nói. Chương trình sẽ bao gồm các bước cơ bản sau:
\begin{figure}[h!]
	\centering
	\includegraphics[width=1\textwidth]{
		nlp.jpg
	}
	\caption[Kiến trúc của một chương trình máy tính giao tiếp với con người thông qua tiếng nói ]{
		Kiến trúc của một chương trình máy tính giao tiếp với con người thông qua tiếng nói \label{pic1.1}
	}
\end{figure}


\begin{itemize}
	\item  Nhận dạng tiếng nói: ở bước này, máy tính sẽ nhận dạng yêu cầu của người dùng ở dạng tiếng nói và chuyển yêu cầu này về dạng văn bản.
	\item Xử lý yêu cầu: máy tính sẽ phân tích yêu cầu ở dạng văn bản, xử lý, đưa ra câu trả lời sử dụng các kỹ thuật trong xử lý văn bản.
	\item Tổng hợp tiếng nói: ở bước này, câu trả lời sẽ được chuyển từ dạng văn bản sang tiếng nói và gửi tới người dùng.
\end{itemize}


\section{Các bài toán cơ bản trong NLP } 
---------------------------------
\subsection{Mô hình hóa ngôn ngữ (Language modelling)}
Mô hình hóa ngôn ngữ (LM) gán một xác suất cho bất kỳ chuỗi từ nào. Về cơ bản, trongbài toán này, ta cần dự đoán từ tiếp theo xuất hiện theo trình tự, dựa trên lịch sử của các từ đã xuất hiện trước đó. LM rất quan trọng trong các ứng dụng khác nhau của NLP, và là lý do tại sao máy móc có thể hiểu được thông tin định tính. Một số ứng dụng của Mô hình hóa ngôn ngữ bao gồm: nhận dạng giọng nói, nhận dạng ký tự quang học, nhận dạng chữ viết tay, dịch máy và sửa lỗi chính tả.\cite{WEBSITE:3}

\subsection{Phân loại văn bản (Text classification)}
Phân loại văn bản gán các danh mục được xác định trước cho văn bản dựa trên nội dung của nó. Cho đến nay, phân loại văn bản là ứng dụng phổ biến nhất của NLP, được sử dụng để xây dựng các công cụ khác nhau như trình phát hiện thư rác và chương trình phân tích cảm xúc.\cite{WEBSITE:3}

\subsection{Trích xuất thông tin (Information extraction)}
Trích xuất thông tin (IE) tự động trích xuất thông tin có liên quan từ các tài liệu văn bản không có cấu trúc và / hoặc bán cấu trúc. Ví dụ về các loại tài liệu này bao gồm lịch sự kiện từ email hoặc tên của những người được đề cập trong một bài đăng trên mạng xã hội.\cite{WEBSITE:3}

\subsection{Truy xuất thông tin (Information retrieval)}
Google là một loại hệ thống Truy xuất Thông tin (IR) phổ biến nhất mà chúng ta thường sử dụng. IR làm nhiệm vụ tìm kiếm các tài liệu có liên quan từ một bộ dữ liệu lớn các tài liệu liên quan đến truy vấn do người dùng thực hiện.\cite{WEBSITE:3}

\subsection{Tác tử phần mềm hội thoại (Conversational agent)}
Tác tử phần mềm hội thoại thuộc AI hội thoại, liên quan đến việc xây dựng các hệ thống đối thoại mô phỏng các tương tác của con người. Các ví dụ phổ biến về AI hội thoại bao gồm Alexa, Siri, Google Home, Cortana, hay trợ lý ảo ViVi. Các công nghệ như chatbot cũng được hỗ trợ bởi  tác tử phần mềm hội thoại và ngày càng phổ biến trong các doanh nghiệp.\cite{WEBSITE:3}

\subsection{Tóm tắt văn bản (Text summarization)}
Tự động tóm tắt là quá trình rút ngắn một tập hợp dữ liệu để tạo một tập hợp con đại diện cho thông tin quan trọng nhất hoặc có liên quan trong nội dung gốc.\cite{WEBSITE:3}

\subsection{Hỏi đáp (Question answering)}
Hỏi đáp là bài toán xây dựng các hệ thống có thể tự động trả lời cho các câu hỏi do con người đặt ra bằng ngôn ngữ tự nhiên.\cite{WEBSITE:3}

\subsection{Dịch máy (Machine translation)}
Dịch máy (MT) là một nhánh con của ngôn ngữ học tính toán liên quan đến việc chuyển đổi một đoạn văn bản từ ngôn ngữ này sang ngôn ngữ khác. Một ứng dụng phổ biến của loại này là Google Dịch.\cite{WEBSITE:3}

\subsection{Mô hình hóa chủ đề (Topic modelling)}
Mô hình hóa chủ đề là một kỹ thuật Học máy không giám sát giúp khám phá cấu trúc chủ đề của một bộ tài liệu lớn. Ứng dụng NLP này là một công cụ khá phổ biến, được sử dụng trên nhiều lĩnh vực khác nhau – như Văn học, và Tin sinh học.\cite{WEBSITE:3}


%----------------------------------------------------------------------------------------

\section{Các công cụ giải quyết các bài toán NLP} 
\subsection{NLTK}
Natural Language ToolKit (NLTK) là một trong những nền tảng hàng đầu để xây dựng các chương trình Python xử lý và phân tích dữ liệu ngôn ngữ của con người. 

NLTK cung cấp giao diện dễ sử dụng cho hơn 50 tài nguyên ngữ liệu và từ vựng như mạng từ, cùng với một bộ thư viện xử lý văn bản để phân loại, mã hóa, tạo gốc, gắn thẻ, phân tích cú pháp và lập luận ngữ nghĩa.\cite{WEBSITE:3}

Ví dụ về sử dụng NLTK để xử lí dữ liệu

\lstinputlisting[style=codePython]{"Code/nltk.py"}

Kết quả:

\lstinputlisting[style=plaintext]{"Code/nltk_result.txt"}

\subsection{Spacy}
Bản phát hành đầu tiên của SpaCy là vào tháng 2 năm 2015, khiến nó trở thành một trong những framework nguồn mở gần đây dành cho các ứng dụng Xử lý ngôn ngữ tự nhiên Python. So với NLTK được tạo ra vào năm 2001, những người sáng tạo SpaCy có đủ thời gian để tìm hiểu NLTK và xem nó còn thiếu ở đâu. Một trong những cải tiến dễ nhận biết nhất so với NTLK bao gồm các cải tiến về hiệu suất, vì SpaCy sử dụng một số thuật toán mới nhất và tốt nhất.

Ngoài ra, SpaCy được ghi chép rất đầy đủ và được thiết kế để hỗ trợ khối lượng lớn dữ liệu. Nó cũng bao gồm một loạt các mô hình Xử lý ngôn ngữ tự nhiên được đào tạo trước, giúp việc học, giảng dạy và thực hành Xử lý ngôn ngữ tự nhiên với SpaCy trở nên dễ tiếp cận hơn \cite{WEBSITE:3}. 

Ví dụ tách chuỗi bằng spacy:

\lstinputlisting[style=codePython]{"Code/spacy.py"}

Ví dụ đọc dữ liệu từ file sau đó xử lí:

\lstinputlisting[style=codePython]{"Code/spacy1.py"}

Kết quả: 

\lstinputlisting[style=plaintext]{"Code/spacy_result.txt"}

\subsection{Stanford CoreNLP}
CoreNLP là một thư viện cực kỳ phổ biến cho các tác vụ Xử lý Ngôn ngữ tự nhiên, được xây dựng bởi cộng đồng NLP Stanford. Ngược lại với NLTK và SpaCy, được viết bằng Python hoặc Cython tương ứng, CoreNLP bằng Java – có nghĩa là máy tính của bạn sẽ cần phải có JDK (nhưng nó có API cho hầu hết các ngôn ngữ lập trình).

Trên trang chủ CoreNLP, các nhà phát triển mô tả CoreNLP là “nơi duy nhất để xử lý ngôn ngữ tự nhiên trong Java! CoreNLP cho phép người dùng lấy các chú thích ngôn ngữ cho văn bản, bao gồm mã thông báo và ranh giới câu, các phần của giọng nói, các thực thể được đặt tên, giá trị số và thời gian, trình phân tích cú pháp phụ thuộc và ý kiến chính, tình cảm, phân bổ trích dẫn và quan hệ. CoreNLP hiện hỗ trợ 6 ngôn ngữ: Ả Rập, Trung Quốc, Anh, Pháp, Đức và Tây Ban Nha.

Một trong những ưu điểm chính của CoreNLP là nó có khả năng mở rộng rất cao, trở thành lựa chọn phù hợp cho các tác vụ phức tạp. Một yếu tố khác là CoreNLP được xây dựng chú trọng đến tốc độ – nó được tối ưu hóa để vận hành cực kỳ nhanh. \cite{WEBSITE:3}

Ví dụ sử dụng Standford CoreNlp:

\lstinputlisting[style=codePython]{"Code/StandfordNLP.py"}

Kết quả:

\lstinputlisting[style=plaintext]{"Code/coreNLP-result.txt"}

\subsection{Gensim}
Gensim là một framework Python mã nguồn mở chuyên dụng, được sử dụng để biểu diễn tài liệu dưới dạng vectơ ngữ nghĩa theo những cách hiệu quả nhất và dễ dàng nhất có thể. Các tác giả đã thiết kế Gensim để xử lý văn bản thô, không có cấu trúc bằng cách sử dụng nhiều thuật toán học máy – vì vậy sử dụng Gensim để tiếp cận các tác vụ như Lập mô hình chủ đề là một ý tưởng tốt. Thêm vào đó, Gensim làm rất tốt việc xác định các điểm tương đồng trong văn bản, lập chỉ mục văn bản và điều hướng các tài liệu khác nhau.\cite{WEBSITE:3}

Gensim được xây dựng vì 3 lý do:

\begin{itemize}
	\item Tính thực tiễn – tập trung vào các thuật toán đã được chứng minh, đã được kiểm chứng để giải quyết các vấn đề thực tế của ngành. Gensim tập trung nhiều hơn vào kỹ thuật, ít hơn về học thuật.
	\item Độc lập đối với bộ nhớ – không cần toàn bộ kho dữ liệu đào tạo phải nằm hoàn toàn trong RAM cùng một lúc. Nó có thể xử lý kho dữ liệu lớn, quy mô web bằng cách sử dụng luồng dữ liệu.
	\item Hiệu suất – triển khai tối ưu hóa cao các thuật toán không gian vectơ phổ biến sử dụng C, BLAS và ánh xạ bộ nhớ
\end{itemize}

Ví dụ sử dụng Gensim:

\lstinputlisting[style=codePython]{"Code/gensim.py"}

Kết quả:

\lstinputlisting[style=plaintext]{"Code/gensim-result.txt"}
%----------------------------------------------------------------------------------------



			
			
			

\chapter{Feature Engineering và các model xử lý dữ liệu dạng văn bản}
 % Tên của chương
 \label{Chapter2}

\section{Các bước xử lý một bài toán học máy (MLP)}
Bất kỳ hệ thống thông minh nào về cơ bản đều bao gồm các bước bắt đầu từ việc nhập dữ liệu thô, sử dụng các kỹ thuật để sắp xếp, xử lý, thiết kế các đặc trưng (feature) và thuộc tính có ý nghĩa từ dữ liệu này. Sau đó, chúng ta thường sử dụng các kỹ thuật như mô hình thống kê hoặc các mô hình học máy để xây dựng các mô hình với mục đích giải quyết yêu cầu đặt ra. Một quy trình tiêu chuẩn điển hình dựa trên mô hình tiêu chuẩn công nghiệp CRISP-DM được mô tả như hình \ref{pic2.1}

\begin{figure}[h!]
	\centering
	\includegraphics[width=1\textwidth]{
		mlp.png
	}
	\caption[Mô hình tiêu chuẩn công nghiệp CRISP-DM]{
		Mô hình tiêu chuẩn công nghiệp CRISP-DM \label{pic2.1}
	}
\end{figure}

\section{Giới thiệu về Feature Engineering}
Feature Engineerning là một giai đoạn không thể thiếu trong quá trình phát triển bất kỳ một hệ thống thông minh nào. Mặc dù hiện nay chúng ta có rất nhiều các phương pháp mới như học sâu, siêu mô hình hỗ trợ học máy tự động (automated machine learning), tuy nhiên với mỗi vấn đề cụ thể cần giải quyết luôn có những đặc trưng quan trọng hơn, có giá trị hơn để quyết định hiệu suất hệ thống của bạn.


\section{Cơ bản về đặc trưng của dữ liệu}

Một đặc trưng (feature) thường là một đại diện cụ thể trên dòng đầu tiên của dữ liệu thô, là một thuộc tính riêng lẻ, có thể đo lường và mô tả bởi một cột trong tập dữ liệu. Lấy ví dụ với một tập dữ liệu hai chiều, mỗi observation (quan sát) được mô tả bởi một hàng và mỗi đặc trưng được mô tả bởi một cột, sẽ có giá trị cụ thể cho mỗi đặc trưng của từng quan sát như hình \ref{pic2.2}

\begin{figure}[h!]
	\centering
	\includegraphics[width=0.5\textwidth]{
		featuresData.png
	}
	\caption[Đặc trưng của dữ liệu]{
		Đặc trưng của dữ liệu \label{pic2.2}
	}
\end{figure}

Như ví dụ ở hình trên, mỗi hàng thường biểu thị một vectơ đặc trưng và tập hợp tất cả các đặc trưng trên tất cả các quan sát tạo thành một ma trận hai chiều còn được gọi là feature-set. Thông thường, các thuật toán học máy hoạt động với các ma trận số hóa hoặc tenxo bởi vậy hầu hết các kỹ thuật feature engineering sẽ xử lý việc chuyển đổi dữ liệu thô thành các dạng biểu diễn số học giúp các thuật toán có thể dễ dàng hiểu được.

Các đặc trưng có thể chia thành hai loại chính:

\begin{itemize}
	\item \textbf{Đặc trưng thô (Raw features)}: là các đặc trưng vốn có được lấy trực tiếp từ tập dữ liệu mà không cần sử dụng thêm thao tác kỹ thuật nào
	\item \textbf{Đặc trưng phát sinh (Derived features)}: là các đặc trưng được thu được sau quá trình feature engineering, là kết quả của quá trình trích xuất và xử lý các đặc trưng có sẵn. 
\end{itemize}

\section{Kĩ thuật tạo đặc trưng dữ liệu}

Về kĩ thuật tạo đặc trưng cho dữ liệu có 3 phương pháp chính:
\begin{itemize}
	\item \textbf{Trích lọc feature}: Không phải toàn bộ thông tin được cung cấp từ một biến dự báo hoàn toàn mang lại giá trị trong việc phân loại. Do đó chúng ta cần phải trích lọc những thông tin chính từ biến đó. Chẳng hạn như trong các mô hình chuỗi thời gian chúng ta thường sử dụng kĩ thuật phân rã thời gian để trích lọc ra các đặc trưng như Ngày thành Năm, Tháng, Quí,…. Các đặc trưng mới sẽ giúp phát hiện các đặc tính chu kì và mùa vụ, những đặc tính mà thường xuất hiện trong các chuỗi thời gian
	\item \textbf{Biến đổi feature}:  Biến đổi dữ liệu gốc thành những dữ liệu phù hợp với mô hình nghiên cứu. Những biến này thường có tương quan cao hơn đối với biến mục tiêu và do đó giúp cải thiện độ chính xác của mô hình
	\item \textbf{Lựa chọn feature} : Phương pháp này được áp dụng trong những trường hợp có rất nhiều dữ liệu mà chúng ta cần lựa chọn ra dữ liệu có ảnh hưởng lớn nhất đến sức mạnh phân loại của mô hình. Các phương pháp có thể áp dụng đó là ranking các biến theo mức độ quan trọng bằng các mô hình như Random Forest, Linear Regression, Neural Network,…; Sử dụng chỉ số IV trong scorecard; Sử dụng các chỉ số khác như AIC hoặc Pearson Correlation, phương sai.
\end{itemize}

\section{Xử lý dữ liệu dạng văn bản}

\begin{enumerate}
	\item Import các thư viện cần thiết và khởi tạo một đoạn văn bản mẫu cần xử lý \lstinputlisting[style=codePython]{"Code/dependencies.py"}
	Kết quả: 
	\lstinputlisting[style=plaintext]{"Code/dependencies.txt"}
	\item
	\item Tiền xử lý văn bản \cite{WEBSITE:10}
	\begin{itemize}
		\item \textbf{Xóa thẻ tags}: Văn bản chúng ta gặp thường chứa nội dung không cần thiết như các thẻ \textbf{HTML}, không có giá trị khi phân tích. Thư viện \textbf{BeautifulSoup} là một công cụ tuyệt vời và cần thiết để xử lý trong trường hợp này.
		\item \textbf{Xóa các ký tự có dấu}:Trong bất kỳ văn bản nào, đặc biệt nếu bạn đang xử lý ngôn ngữ tiếng Anh, thường các bạn cần phải xử lý các ký tự có dấu. Do đó, chúng ta vần đảm bảo rằng các ký tự này cần được chuyển đổi và chuẩn hóa thành các ký tự ASCII
		\item \textbf{Biến đổi các từ viết tắt}: Trong tiếng Anh, các từ viết tắt về cơ bản là phiên bản rút gọn của các từ hoặc âm tiết. Những từ viết tắt của các từ hoặc cụm từ thường được tạo ra bằng cách loại bỏ các chữ cái và âm tiết. Ví dụ như: \textbf{do not -> don't, I would -> I'd}. Chuyển đổi từ dạng viết tắt thành dạng đầy đủ cũng là một bước cần thiết để chuẩn hóa văn bản.
		\item \textbf{Xóa các ký tự đặc biệt}: Các ký tự đặc biệt thường là các ký tự không phải là chữ và số, thường gây "nhiễu" cho dữ liệu của chúng ta. Thông thường, regular expressions \textbf{(regexes)} có thể được sử dụng để xử lý vấn đề này.
		\item \textbf{Từ gốc và ngữ pháp}:  Trong các ngữ cảnh khác nhau, các từ gốc thường được gắn thêm các tiền tố và hậu tố vào để đúng với ngữ pháp. Ví dụ các từ: \textbf{WATCHES, WATCHING}, and \textbf{WATCHED}. Chúng ta có thể thấy rằng chúng đều có chung từ gốc là \textbf{WATCH}.
		\item \textbf{Xóa các stopwords}: stopwords là các từ có ít hoặc không có ý nghĩa gì đặc biệt khi xây dựng các đặc trưng
	\end{itemize}
	\lstinputlisting[style=codePython]{"Code/simpleDataProcessing.py"}
	
	Kết quả:
	
	\lstinputlisting[style=plaintext]{"Code/simpleDataProcessing.txt"}

\end{enumerate}

\section{Các model phổ biến trong việc xử lý dữ liệu dạng văn bản}

\subsection{Bag of Words Model - Túi từ}
Mô hình Bag of words biểu diễn cho mỗi mẫu dữ liệu văn bản dưới dạng một vecto số trong đó mỗi chiều là một từ cụ thể trong kho dữ liệu và giá trị có thể là tần số của nó xuất hiện trong đoạn văn bản (giá trị có thể là 0 hoặc 1) hoặc thậm chí là các giá trị có trọng số. \cite{WEBSITE:10}

Tên mô hình này là Bag of words thể hiện theo đúng nghĩa đen của nó nghĩa là một túi các từ, không quan tâm đến trật tự, trình tự, ngữ pháp. \cite{WEBSITE:10}

Ví dụ về model \textbf{Bag of Words}

\lstinputlisting[style=codePython]{"Code/bagOfWords.py"}

Kết quả
\lstinputlisting[style=plaintext]{"Code/bagOfWords.txt"}

\subsection{Bag of N-Grams}
Là phân bố xác suất trên các tập văn bản\\

Cho biết xác suất của 1 câu (hoặc 1 cụm từ) thuộc 1 ngôn ngữ là bao nhiêu
Mô hình ngôn ngữ tốt sẽ đánh giá đúng các câu đúng ngữ pháp, trôi chảy hơn các cụm từ có thứ tự ngẫu nhiên \cite{NGRAM1}

Các mô hình N-grams phổ biến \cite{NGRAM}:
\begin{itemize}
	\item \textbf{Unigram}: mô hình với n=1, tức là ta sẽ tính tần suất xuất hiện của một kí tự (từ), như: “k”, “a”,…
	\item \textbf{Bigrams}: với n=2 , là mô hình được sử dụng nhiều trong việc phân tích các hình thái cho ngôn ngữ
	\item \textbf{Trigrams}:với n-3, với n càng lớn thì độ chính xác càng cao tuy nhiên đi kèm với đó thì độ phức tạp cũng lớn hơn
\end{itemize}

Mô hình N-gram \cite{NGRAM}:
Mục tiêu: Tính xác suất của 1 câu hoặc 1 cụm từ:
\begin{itemize}
	\item Để tính xác suất của một câu: W1W2...Wk...Wn, Theo công thứ Bayes \cite{NGRAM} \ref{2.1}:
	\begin{equation}
		p(w_1...w_n) = \frac{count(w_1...w_n)}{N} \label{2.1} 
	\end{equation}
	
	\item Tuy nhiên, công thức trên có độ phức tạp lớn, vì vậy người ta thường sử dụng công thức Markov \cite{NGRAM} \ref{2.2}:
	\begin{equation}
		p(w_1...w_n) = p(w_1) * p(w_2|w_1) * p(w_3|w_1w_2)  *...*p(w_n|w_1...w_(n-1)) \label{2.2} 
	\end{equation}
\end{itemize}

Ví dụ về model \textbf{Bag of N-grams}:

\lstinputlisting[style=codePython]{"Code/nGram.py"}

Kết quả \ref{pic2.3}: 
\begin{figure}[h!]
	\centering
	\includegraphics[width=1\textwidth]{
			ngramFigures.png
		}
	\caption[Kết quả của model Bag of N-grams]{
			Kết quả của model Bag of N-grams  \cite{WEBSITE:11} \label{pic2.3}
		}
\end{figure}

\subsection{TF-IDF - Term Frequency-Inverse Document Frequency}
TF-IDF là viết tắt của Term Frequency-Inverse Document Frequency.\\
Hiểu một cách đơn giản nó là sự kết hợp của tần số xuất hiện của một từ trong một mẫu và nghịch đảo của tần số của từ đó trong toàn bộ tập dữ liệu\\
Kỹ thuật này được phát triển để đánh giá kết quả cho các truy vấn trong công cụ tìm kiếm và hiện tại nó là một phần không thể thiếu trong xử lý ngôn ngữ tự nhiên \cite{WEBSITE:11}\\
Về mặt toán học có thể định nghĩa như sau \cite{WEBSITE:10} \ref{2.3}:

\begin{equation}
	tfidf(w,D) = tf(w,D)*idf(w,D)=tf(w,D)*log(\frac{C}{df(w)}) \label{2.3}
\end{equation}

Theo công thức trên, \textbf{tfidf(w,D)} là một 'score'. TF-IDF cho từ \textbf{w} trong mẫu \textbf{D}. Thuật ngữ \textbf{tf(w,D)} đại diện cho tần số của từ \textbf{w} xuất hiện trong mẫu \textbf{D} có thể lấy được từ mô hình \textbf{Bag of words}. Thuật ngữ \textbf{idf(w,D)} là tần số nghịch đảo của \textbf{w} có thể tính là \textbf{log} của tổng số mẫu dữ liệu xuất hiện từ \textbf{w}. \\
Mô hình này có thể có rất nhiều biến thể khác nhau, tuy nhiên chúng đều cho kết quả khá giống nhau. 

Ví dụ về model \textbf{TF-IDF}: 

\lstinputlisting[style=codePython]{"Code/tfidf.py"}

Kết quả \ref{pic2.4}:

\begin{figure}[h!]
	\centering
	\includegraphics[width=1\textwidth]{
		tfidfFigures.png
	}
	\caption[Kết quả của model TF-IDF]{
		Kết quả của model TF-IDF   \cite{WEBSITE:11} \label{pic2.4}
	}
\end{figure}

\subsection{Document Similarity}
Document Similarity (hay độ tương tự của văn bản) là quá trình sử dụng số liệu dựa trên khoảng cách hoặc độ tương tự có thể sử dụng để xác định mức độ tương đương của một văn bản với bất kỳ văn bản nào khác dựa trên các đặc trưng được trích xuất ra từ \textbf{bag of words} hoặc \textbf{tf-idf}.\\
Sự tương tự giữa các mẫu dữ liệu trong một kho văn bản cũng được hiểu là sự tương tự giữa từng cặp mẫu trong toàn bộ kho văn bản đó

Có rất nhiều công thức có thể sử dụng để tính toán độ tương tự này như khoảng cách cosin, khoảng cách euclide, khoảng cách manhattan...  \cite{WEBSITE:11}

Ví dụ sử dụng \textbf{cosine} để tính toán độ tương tự 
\lstinputlisting[style=codePython]{"Code/cosine.py"}

Kết quả \ref{pic2.5}:
\begin{figure}[h!]
	\centering
	\includegraphics[width=1\textwidth]{
		cosine.png
	}
	\caption[Kết quả của model  Document Similarity.]{
		Kết quả của model  Document Similarity.   \cite{WEBSITE:11} \label{pic2.5}
	}
\end{figure}

Về cơ bản, khoảng cách cosin cung cấp cho chúng ta một số liệu biểu thị góc giữa 2 vecto đặc trưng tương ứng của từng mẫu. Góc giữa hai mẫu càng gần nhau thì độ tương tự của hai mẫu đó càng lớn như được mô tả trong hình dưới đây\\

\begin{figure}[h!]
	\centering
	\includegraphics[width=1\textwidth]{
		cosin1.png
	}
	\caption[Độ tương tự của hai mẫu theo khoảng cách cosine]{
		Độ tương tự của hai mẫu theo khoảng cách cosine.   \cite{WEBSITE:11} \label{pic2.5}
	}
\end{figure}
%
%\section{Trích lọc đặc trưng cho dữ liệu dạng văn bản}
%
%Dữ liệu văn bản có thể đến từ nhiều nguồn và nhiều định dạng khác nhau (kí tự thường, kí tự hoa, kí tự đặc biệt,…). Có nhiều phương pháp xử lý dữ liệu phù hợp với từng đề tài cụ thể. Ở đây chúng tôi sẽ sử dụng kĩ thuật mã hóa (tokenization)\cite{WEBSITE:3}
%
%Mã hóa đơn giản là việc chúng ta chia đoạn văn thành các câu văn, các câu văn thành các từ. 
%
%Trong mã hóa thì từ là đơn vị cơ sở. Chúng ta cần một bộ tokenizer có kích thước bằng toàn bộ các từ xuất hiện trong văn bản hoặc bằng toàn bộ các từ có trong từ điển. 
%
%Một câu văn sẽ được biểu diễn bằng một sparse vector mà mỗi một phần tử đại diện cho một từ, giá trị của nó bằng 0 hoặc 1 tương ứng với từ không xuất hiện hoặc có xuất hiện.
%
%Chúng ta sử dụng các túi từ (bags of words) để tạo ra một vector có độ dài bằng độ dài của tokenizer và mỗi phần tử của túi từ sẽ đếm số lần xuất hiện của một từ trong câu và sắp xếp chúng theo một vị trí phù hợp trong vector. Bên dưới là code minh họa cho quá trình này.
%
%\lstinputlisting[style=codePython]{"Code/tokenization.py"}
%
%Quá trình này có thể được mô tả như hình \ref{pic2.3}:
%\begin{figure}[h!]
%	\centering
%	\includegraphics[width=1\textwidth]{
%		tokenization.png
%	}
%	\caption[Biểu diễn kĩ thuật mã hóa]{
%		Biểu diễn kĩ thuật mã  \label{pic2.3}
%	}
%\end{figure}
%
%Cách biểu diễn theo túi từ có hạn chế đó là chúng ta không phân biệt được 2 câu văn có cùng các từ bởi túi từ không phân biệt thứ tự trước sau của các từ trong một câu. Chặng như ‘you have no dog’ và ‘no, you have dog’ là 2 câu văn có biểu diễn giống nhau mặc dù có ý nghĩa trái ngược nhau. Chính vì thế phương pháp N-gram sẽ được sử dụng thay thế.
%
%\lstinputlisting[style=codePython]{"Code/nGram.py"}
%
%
%Những từ hiếm khi được tìm thấy trong tập văn bản (corpus) nhưng có mặt trong một văn bản cụ thể có thể quan trọng hơn. Do đó cần tăng trọng số của các nhóm từ ngữ để tách chúng ra khỏi các từ phổ biến. Cách tiếp cận này được gọi là TF-IDF (Term Frequency - Inverse Document Frequency)
%
%Các chỉ số chính đánh giá tần xuất xuất hiện của một từ trong toàn bộ tập văn bản là idf và tfidf được tính theo công thức \ref{tdf} \ref{tfidf}:
%
%\begin{equation}
%	idf(t,D) = log(\frac{|D|}{df(d,t) + 1})\\ \label{tdf}
%\end{equation}
%
%\begin{equation}
%	tfidf(t,d,D) = tf(d,d) x idf(i, D) \label{tfidf}
%\end{equation}
%
%Ở đây:
%\begin{itemize}
%	\item \textbf{|D|}: là số lượng các văn bản trong tập văn bản
%	\item \textbf{df(d,t)}: là số lượng các văn bản là từ t xuất hiện
%	\item \textbf{tf(dt)}: là tần suất các từ xuất hiện trong một văn bản
%\end{itemize}
%Như vậy một từ càng phổ biến khi idf càng nhỏ và tfidf càng lớn.
%Do đó để tính \textbf{tfidf} cho các từ trong văn bản, ta có thể làm như sau:
%
%\lstinputlisting[style=codePython]{"Code/tfidf.py"}
%
%Kết quả:
%
%\lstinputlisting[style=plaintext]{"Code/tfidf.txt"}
%
%Ta có thể thấy từ "I" xuất hiện ở toàn bộ các câu và không mang nhiều ý nghĩa của chủ đề của câu nên có thể coi là một stopword. Bằng phương pháp lọc cận trên của tần suất xuất hiện từ trong văn bản là 90\% ta đã loại bỏ được từ này khỏi dictionary. 
\chapter{Topic Modeling}
\label{Chapter3}
\section{Topic Modeling là gì ?}
\begin{figure}[h!]
	\centering
	\includegraphics[width=1\textwidth]{
		topicModeling.png
	}
	\caption[Topic Modeling]{
		Topic Modeling   \cite{WEBSITE:11}
	}
\end{figure}
Topic modeling hay Mô hình hóa chủ đề là một kĩ thuật học máy tự động phân tích dữ liệu văn bản để xác định các từ cụm cho một tập hợp các tài liệu. Điều này được gọi là học máy 'không giám sát' vì nó không yêu cầu dữ liệu  được con người phân loại trước đây.

Ý tưởng về các topic models xoay quanh quá trình sắp xếp các văn bản vào những dạng chủ đề, khái niệm. 

Mỗi chủ để được biểu diễn dưới dạng là tập hợp của các từ/thuật ngữ có trong kho văn bản.

Nếu chúng xuất hiện cùng với nhau các từ/thuật ngữ này sẽ biểu thị, tượng trưng cho một chủ đề hoặc khái niệm cụ thể. Chúng ta có thể dễ dạng phân biệt các chủ đề với nhau nhờ ngữ nghĩa của các thuật ngữ đó

Tuy nhiên, các chủ đề thường có sự chồng chéo nhất định trên dữ liệu. Các topic models sẽ cực kỳ hữu ích trong việc tóm tắt, rút gọn khối lượng lớn tài liệu, văn bản để trích xuất, mô tả các khái niệm chính nhất. Chúng cũng hữu ích trong việc trích xuất các đặc trưng từ dữ liệu văn bản để nắm bắt các pattern tiềm ẩn trong dữ liệu đó. \cite{WEBSITE:12}



\section{Các kỹ thuật sửu dụng trong Topic Modeling}

Có nhiều kỹ thuật khác nhau để mô hình hóa chủ đề văn bản và hầu hết chúng liên quan đến một số dạng phân tách ma trận

Một số kỹ thuật như \cite{WEBSITE:11}:

\begin{enumerate}
	\item \textbf{Latent Semantic Analysis  (LSA) - Phân tích ngữ nghĩa tiềm ẩn} :sử dụng các phương pháp phân tách ma trận, cụ thể hơn là phân tách các giá trị đơn lẻ.
	\item \textbf{Latent Dirichlet Allocation (LDA) - Phân bố Dirichlet tiền ẩn}: một kỹ thuật sử dụng mô hình xác suất tổng quát trong đó mỗi mẫu tài liệu bao gồm một sự kết hợp của một số chủ đề và mỗi thuật ngữ hoặc từ có thể được gán cho một chủ đề cụ thể.
\end{enumerate}

\subsection{Latent Semantic Analysis (LSA)}

LSA (Phân tích ngữ nghĩa tiềm ẩn) còn được gọi là LSI (Chỉ số ngữ nghĩa tiềm ẩn) LSA sử dụng mô hình túi từ (BoW). 
Hàng và cột đại diện cho tài liệu. \\
LSA học các chủ đề tiềm ẩn bằng cáchsử dụng các phương pháp phân tách ma trận, cụ thể hơn là phân tách các giá trị đơn lẻ.

\textbf{Singular Value Decomposition(SVD)}
SVD là một phương pháp phân tích ma trận thành nhân tử đại diện cho một ma trận trong tích của hai ma trận \ref{equa1} \cite{WEBSITE:13}

\begin{equation}
	M=U \sum V^* \label{equa1}
\end{equation}
Trong đó:
\begin{itemize}
	\item M: là ma trận cỡ mxm\\
	\item U: là ma trận cỡ mxn phía bên trái\\
	\item V: là ma trận cỡ mxn phía bên phải\\
	\item V*: là ma trận chuyển vị của ma trận V\\
\end{itemize}

\textbf{Thực nghiệm thuật LSA sử dụng Gensim} \cite{WEBSITE:13}
\begin{enumerate}
	\item \textbf{Import các thư viện cần thiết}
		\lstinputlisting[style=codePython]{"Code/dependencies1.py"}
	\item \textbf{Load dữ liệu mẫu}:
		\lstinputlisting[style=codePython]{"Code/loadData.py"}
	
	\item \textbf{Tiền xử lý dữ liệu}:
		\lstinputlisting[style=codePython]{"Code/preprocessing.py"}
		
	\item \textbf{Chuẩn bị các dữ liệu dạng văn bản}:
		\lstinputlisting[style=codePython]{"Code/prepareCorpus.py"}
		
	\item \textbf{Khởi tạo model LSA sử dụng Gensim}:
		\lstinputlisting[style=codePython]{"Code/lsaGensim.py"}
		
	\item \textbf{Xác định số lượng topics}: 
		\lstinputlisting[style=codePython]{"Code/determineTopics.py"}
	\item \textbf{Kết quả thực nghiệm} :
		\lstinputlisting[style=plaintext]{"Code/resultLSA.txt"}
		\begin{figure}[h!]
			\centering
			\includegraphics[width=1\textwidth]{
				resultLSA.png
			}
			\caption[Kết quả thuật toán LSA trong Topic Modeling]{
				Kết quả thuật toán LSA trong Topic Modeling 
			}
		\end{figure}
\end{enumerate}

\subsection{Latent Dirichlet Allocation (LDA)}
Model LDA là lớp mô hình sinh (generative model) cho phép xác định một tợp hợp các chủ đề tưởng tượng (imaginary topics) mà mỗi topic sẽ được biểu diễn bởi tập hợp các từ. Mục tiêu của LDA là mapping toàn bộ các văn bản sang các topics tương ứng sao cho các từ trong mỗi một văn bản sẽ thể hiện những topic tưởng tượng ấy.

\textbf{Ứng dụng của gensim trong bài toán LDA} \cite{WEBSITE:14}

Dữ liệu được sử dụng là 20-newgroups dataset

Dữ liệu bao gồm 11k các posts liên quan đến 20 chủ đề khác nhau đã được gán nhãn.

\begin{enumerate}
	\item \textbf{Import các thư viện cần thiết}
	\lstinputlisting[style=codePython]{"Code/importData.py"}
	
	\item \textbf{Dữ liệu mẫu} %\ref{tab:data}:
	\lstinputlisting[style=plaintext]{"Code/loadDataRes.txt"}
	
%	\begin{table}[h!]
%		\caption{Dữ liệu mẫu: }
%		\label{tab:data}
%		\centering
%		\begin{tabular}{l l l 1}
%			\toprule
%			\tabhead{No}&\tabhead{content}&\tabhead{target}&\tabhead{target_names}\\
%			\midrule
%			0 & From: lerxst@wam.umd.edu (where's my thing)... & 7 & rec.autos\\
%			1 & From: guykuo@carson.u.washington.edu (Guy Kuo)... & 4 & comp.sys.mac.hardware\\
%			10 & From: guykuo@carson.u.washington.edu (Guy Kuo)... & 8 & rec.motorcycles\\
%			100 & From: guykuo@carson.u.washington.edu (Guy Kuo)... & 6 & misc.forsale\\
%			1000 & 	From: dabl2@nlm.nih.gov (Don A.B. Lindbergh)... & 2 & comp.os.ms-windows.misc\\
%			\bottomrule \\
%		\end{tabular}
%	\end{table}
	
	\item \textbf{Visualization số lượng các topics}:
		\begin{figure}[h!]
			\centering
			\includegraphics[width=1\textwidth]{
				output.png
			}
			\caption[Visualization số lượng các topics]{
				Visualization số lượng các topics
			}
		\end{figure}
		Như vậy có 20 nhóm, mỗi nhóm có số lượng các posts trong khoảng từ 400-600. phân về các chủ đề như: auto, mobile, medicine,….
	\newpage
	\item \textbf{Tiền xử lý dữ liệu}:
	\lstinputlisting[style=codePython]{"Code/preprocessing1.py"}
	\textbf{Kết quả}:
	\lstinputlisting[style=plaintext]{"Code/preprocessing1Res.txt"}
	
	\item \textbf{Tạo ra các bigram và trigram}:
	Hiện tại các từ vựng đang gồm toàn bộ là những từ đơn. Để tăng độ chính xác cho mô hình ta sẽ cần gom cụm các từ đơn có tần xuất xuất hiện cùng nhau chung thành những collocations có độ dài gồm 2 hoặc 3 từ. Ta sẽ gọi chúng là các bigram hoặc trigram
	\lstinputlisting[style=codePython]{"Code/bgramTrigramModel.py"}
	\textbf{Kết quả }: 
	\lstinputlisting[style=plaintext]{"Code/bgramTrigramModel.txt"}
	
	\item \textbf{Lọai bỏ các stopwords}: Loại bỏ các từ stopwords và chỉ lọc ra các từ vựng là các từ có tag từ loại là \textbf{[‘NOUN’, ‘ADJ’, ‘VERB’, ‘ADV’]}.
	\lstinputlisting[style=codePython]{"Code/deleteStopwords.py"}
	\textbf{Kết quả =}:
	\lstinputlisting[style=plaintext]{"Code/deleteStopwords.txt"}
	
	\item \textbf{Tạo các dictionary}: Từ điển (dictionary) và bộ văn bản (corpus) là 2 input chính cho model LDA
	\lstinputlisting[style=codePython]{"Code/dictionary.py"}
	
	\textbf{Kết quả }:
	\lstinputlisting[style=plaintext]{"Code/dictionary.txt"}
	
	Sau khi xử lý ta đã thu được 1 corpus là list các cặp (index, frequency) mã hóa các văn bản về index được qui định trong dictionary kèm theo tần suất xuất hiện của chúng trong văn bản. Để convert ngược lại từ index sang từ vựng ta sử dụng dictionary là id2word như sau.
	\lstinputlisting[style=codePython]{"Code/convert.py"}
	\textbf{Kết quả }:
	\lstinputlisting[style=plaintext]{"Code/convert.txt"}
	
	\item \textbf{Xây dựng model LDA và in ra các keywords tìm được}:
	\lstinputlisting[style=codePython]{"Code/buildModelLDA.py"}
	\textbf{Kết quả }:
	\lstinputlisting[style=plaintext]{"Code/ldaRes.txt"}
	Đối với topic 1 ta thấy biểu diễn của chúng là: \textbf{'0.019*"information" + 0.017*"file" + 0.015*"program" + 0.015*"include" + '
		'0.013*"system" + 0.013*"also" + 0.013*"available" + 0.011*"software" + '
		'0.011*"standard" + 0.010*"new"} có nghĩa rằng có 10 từ vựng quan trọng nhất đóng góp vào topic này bao gồm: 'imformation', 'file, 'program, 'include, 'system, 'also, 'available, 'software', 'standard', 'new'. Dựa vào cảm quan ta có thể biết được rằng topic này liên quan đến \textbf{CÔNG NGHỆ}.
	
	\item \textbf{Tìm ra topic chính của document}
	\lstinputlisting[style=codePython]{"Code/mainTopics.py"}
	\textbf{Kết quả}: \ref{mainTopics}
	\begin{figure}[h!]
		\centering
		\includegraphics[width=1\textwidth]{
			mainTopics.png
		}
		\caption[Các topic chính của document sau khi được train bằng model LDA]{
			Các topic chính của document sau khi được train bằng model LDA  \label{mainTopics} \cite{WEBSITE:14}
		}
	\end{figure}
\end{enumerate}

\section{Kết luận}
Với bài toán Topic Modeling, việc xuất ra hết những từ/tài liệu quan trọng trong mỗi chủ đề không phải là điều tối ưu, đặc biệt với bộ dữ liệu lớn. Do vậy, giải pháp là ta chỉ hiển thị những từ/tài liệu thuộc top trong chủ đề. Điều này sẽ hữu ích nhất trong việc tìm được tên chủ đề. \cite{WEBSITE:15}

\chapter{Các model nâng cao trong xử lý dữ liệu dạng văn bản}
\label{Chapter4}
\section{Tại sao cần các phương pháp này ?}

Các phương pháp trích xuất đặc trưng truyền thống (dựa trên số lượng) cho dữ liệu văn bản liên quan đến họ phương pháp rất phổ biến là Bag of Words hay bao gồm cả tần số như TF-IDF, N-gram... 

Mặc dù chúng đều là các phương pháp hiệu quả để trích xuất đặc trưng từ văn bản, nhưng do bản chất vốn có của các mô hình đó chỉ là xét trên các từ không có cấu trúc, như vậy chúng ta sẽ mất các thông tin bổ sung như ngữ nghĩa, cấu trúc, trình tự và ngữ cảnh xung quanh các từ gần đó mỗi tài liệu văn bản

Các mô hình nâng cao này có thể trích xuất được các thông tin sâu hơn có thể bao quát được cho cả các từ, thường được gọi là word embeddings.

\section{Mô hình Word2Vec}
Mô hình này được Google rạo ra vào năm 2013 và là mô hình sử dụng deep learning để tính toán và tạo ra các vectơ biểu diễn các từ và bao gồm được cả các tương đồng về ngữ cảnh và ngữ nghĩa của từ đó.

Về cơ bản, đây là mô hình học không giám sát, có thể áp dụng được cho những tập văn bản lớn, tạo ra vốn từ vựng và tạo ra embedding trong không gian vecto cho mỗi từ vựng đó.

Thông thường kích thước của vectơ embeddings và tổng số vectơ là kích thước của không gian từ vựng. Điều này làm cho số chiều của không gian vectơ này thấp hơn rất nhiều so với không gian vectơ được tạo ra bởi mô hình \textbf{Bag of Words} truyền thống.

Có hai kiến trúc khác nhau có thể sử dụng để tạo ra các vectơ embedding này bao gồm \cite{WEBSITE:17} \cite{WEBSITE:18}:
\begin{itemize}
	\item Mô hình \textbf{Continous Bag of Words (CBOW)}
	\item Mô hình \textbf{Skip-gram}
\end{itemize} 


\section{Mô hình Continuous Bag of Words (CBOW)}

Mô hình CBOW sẽ cố gắng dự đoán từ trung tâm (center word hoặc target word) dựa trên ngữ cảnh được tạo ra từ các từ xung quanh nó (surrounding words)

Chúng ta hãy xem xét một câu đơn giản \textbf{"the quick brown fox jumps over the lazy dog"}, chúng ta có thể có các cặp \textbf{ (context-window, target-word)} nếu chọn \textbf{context-window = 2} ta sẽ có \textbf{ ([quick, fox], brown), ([the, brown], quick), ([the, dog], lazy)}. Như vậy ta có thể dự đoán \textbf{target-word} dựa trên \textbf{context-window} như sau \ref{CBOW} \cite{WEBSITE:19}.


\begin{figure}[h!]
	\centering
	\includegraphics[width=0.8\textwidth]{
		CBOW.png
	}
	\caption[Mô hình CBOW]{
		Mô hình CBOW \label{CBOW}
	}
\end{figure}

\section{Xây dựng mô hình Continuous Bag of Words (CBOW)}

\textbf{ Việc xây dựng mô hình sẽ tập trung vào bốn bước sau}:
\begin{enumerate}
	\item \textbf{Xây dựng tập từ vựng}
	\item \textbf{Xây dựng CBOW generator (bao gồm các cặp [context-window, target-word])}
	\item \textbf{Xây dựng kiến trúc mô hình CBOW}
	\item \textbf{Huấn luyện mô hình}
	\item \textbf{Thu được embedding của các từ}			
\end{enumerate}

\subsection{Xây dựng tập từ vựng}
\lstinputlisting[style=codePython]{"Code/buildVocab.py"}
\textbf{Kết quả}:
\lstinputlisting[style=plaintext]{"Code/vocab.txt"}

\subsection{Xây dựng CBOW generator (bao gồm các cặp [context-window, target-word]}
\lstinputlisting[style=codePython]{"Code/buildCBOWgenerator.py"}
\textbf{Kết quả}:
\lstinputlisting[style=plaintext]{"Code/pairGenerator.txt"}

\subsection{Xây dựng kiến trúc mô hình CBOW}
\lstinputlisting[style=codePython]{"Code/buildCBOWDeep.py"}

\subsection{Huấn luyện mô hình}
\lstinputlisting[style=codePython]{"Code/TrainCBOW.py"}
\textbf{Kết quả}:
\lstinputlisting[style=plaintext]{"Code/CBOWRes.txt"}

\subsection{Thu về Embedding Words}
\lstinputlisting[style=codePython]{"Code/getWordsEmbeddings.py"}
\textbf{Kết quả}: \ref{CBOWRes}
\begin{figure}[h!]
	\centering
	\includegraphics[width=1\textwidth]{
		CBOWRes.png
	}
	\caption[Embedding Words]{
		Embedding Words \label{CBOWRes}
	}
\end{figure}



 
%\chapter{Kết luận và đánh giá quá trình tìm hiểu của bản thân}

\section{Kết luận}
 


%----------------------------------------------------------------------------------------
%	(KHÔNG CHỈNH SỬA PHẦN NÀY)
%
%	PHẦN 13: TÀI LIỆU THAM KHẢO
%----------------------------------------------------------------------------------------

\begin{spacing}{1.15}
	\printbibliography[heading=bibintoc, title=Tài liệu tham khảo] % In ra tài liệu tham khảo
\end{spacing}


%----------------------------------------------------------------------------------------
%	PHẦN 14: PHỤ LỤC (THESIS CONTENT - APPENDICES)
%----------------------------------------------------------------------------------------

\appendix % Nói với LaTeX rằng những chương về sau được tính là phụ lục

% Hãy thêm những phụ lục (appendix) của khóa luận/tiểu luận vào thư mục Appendices
% Hãy bỏ chú thích những dòng nếu bạn đã bổ sung những phụ lục vào

%\include{Appendices/AppendixA}
% Phụ lục B

\chapter{Liệt kê source code} % Tên của phụ lục

\label{AppendixB} % Để trích dẫn chương này ở chỗ nào đó trong bài, hãy sử dụng lệnh \ref{AppendixB} 

%----------------------------------------------------------------------------------------

%\section{Ví dụ liệt kê code ngôn ngữ C/C++}
%
%Code tính khoảng thời gian giữa hai thời điểm cho trước. Lệnh thực hiện là:
%\begin{Verbatim}
%\lstinputlisting{"Code/TimeDiff.cpp"}
%\end{Verbatim}
%
%File \file{TimeDiff.cpp}:
%%https://www.programiz.com/cpp-programming/examples/time-structure
%\lstinputlisting{"Code/TimeDiff.cpp"}


%----------------------------------------------------------------------------------------

%\section{Ví dụ liệt kê thông tin ở terminal (console/command prompt) từ file text}
%
%Sau khi biên dịch và chạy file \file{TimeDiff.cpp}, kết quả chạy được hiển thị ở terminal. Trong trường hợp bạn muốn liệt kê quá trình chạy, bạn có thể copy đoạn text ở terminal vào một file text, và liệt kê chúng chẳng hạn như:
%\begin{Verbatim}
%\lstinputlisting[style=console]{"Code/TimeDiff.txt"}
%\end{Verbatim}
%
%File \file{TimeDiff.txt}:
%\lstinputlisting[style=console]{"Code/TimeDiff.txt"}


%----------------------------------------------------------------------------------------

\section{Thuật toán LSA}

\lstinputlisting[style=codePython]{"Code/LSA.py"}
\newpage

\section{Thuật toán LDA}

\lstinputlisting[style=codePython]{"Code/lda.py"}

\newpage
\section{Model CBOW}
\lstinputlisting[style=codePython]{"Code/CBOW.py"}
%----------------------------------------------------------------------------------------

%\section{Ví dụ liệt kê code ngôn ngữ Matlab}
%
%Code biểu diễn bản chất và sai số của phương pháp Euler và Heun trong việc giải phương trình vi phân. Lệnh thực hiện là:
%\begin{Verbatim}
%\lstinputlisting[style=codeMatlab]{"Code/EulerVisualization.m"}
%\end{Verbatim}
%
%File \file{EulerVisualization.m}:
%\lstinputlisting[style=codeMatlab]{"Code/EulerVisualization.m"}


%----------------------------------------------------------------------------------------

%\section{Ví dụ liệt kê file text thông thường (plain text)}
%
%Một file text lưu giữ thông số của một lần chạy mô phỏng động học phân tử. Lệnh thực hiện là:
%\begin{Verbatim}
%\lstinputlisting[style=plaintext]{"Code/minim.mdp"}
%\end{Verbatim}
%
%File \file{minim.mdp}:
%\lstinputlisting[style=plaintext]{"Code/minim.mdp"}




%\include{Appendices/AppendixC}

%----------------------------------------------------------------------------------------

\end{document}  
